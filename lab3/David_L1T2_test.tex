% Options for packages loaded elsewhere
\PassOptionsToPackage{unicode}{hyperref}
\PassOptionsToPackage{hyphens}{url}
%
\documentclass[
]{article}
\usepackage{amsmath,amssymb}
\usepackage{lmodern}
\usepackage{iftex}
\ifPDFTeX
  \usepackage[T1]{fontenc}
  \usepackage[utf8]{inputenc}
  \usepackage{textcomp} % provide euro and other symbols
\else % if luatex or xetex
  \usepackage{unicode-math}
  \defaultfontfeatures{Scale=MatchLowercase}
  \defaultfontfeatures[\rmfamily]{Ligatures=TeX,Scale=1}
\fi
% Use upquote if available, for straight quotes in verbatim environments
\IfFileExists{upquote.sty}{\usepackage{upquote}}{}
\IfFileExists{microtype.sty}{% use microtype if available
  \usepackage[]{microtype}
  \UseMicrotypeSet[protrusion]{basicmath} % disable protrusion for tt fonts
}{}
\makeatletter
\@ifundefined{KOMAClassName}{% if non-KOMA class
  \IfFileExists{parskip.sty}{%
    \usepackage{parskip}
  }{% else
    \setlength{\parindent}{0pt}
    \setlength{\parskip}{6pt plus 2pt minus 1pt}}
}{% if KOMA class
  \KOMAoptions{parskip=half}}
\makeatother
\usepackage{xcolor}
\usepackage[margin=1in]{geometry}
\usepackage{color}
\usepackage{fancyvrb}
\newcommand{\VerbBar}{|}
\newcommand{\VERB}{\Verb[commandchars=\\\{\}]}
\DefineVerbatimEnvironment{Highlighting}{Verbatim}{commandchars=\\\{\}}
% Add ',fontsize=\small' for more characters per line
\usepackage{framed}
\definecolor{shadecolor}{RGB}{248,248,248}
\newenvironment{Shaded}{\begin{snugshade}}{\end{snugshade}}
\newcommand{\AlertTok}[1]{\textcolor[rgb]{0.94,0.16,0.16}{#1}}
\newcommand{\AnnotationTok}[1]{\textcolor[rgb]{0.56,0.35,0.01}{\textbf{\textit{#1}}}}
\newcommand{\AttributeTok}[1]{\textcolor[rgb]{0.77,0.63,0.00}{#1}}
\newcommand{\BaseNTok}[1]{\textcolor[rgb]{0.00,0.00,0.81}{#1}}
\newcommand{\BuiltInTok}[1]{#1}
\newcommand{\CharTok}[1]{\textcolor[rgb]{0.31,0.60,0.02}{#1}}
\newcommand{\CommentTok}[1]{\textcolor[rgb]{0.56,0.35,0.01}{\textit{#1}}}
\newcommand{\CommentVarTok}[1]{\textcolor[rgb]{0.56,0.35,0.01}{\textbf{\textit{#1}}}}
\newcommand{\ConstantTok}[1]{\textcolor[rgb]{0.00,0.00,0.00}{#1}}
\newcommand{\ControlFlowTok}[1]{\textcolor[rgb]{0.13,0.29,0.53}{\textbf{#1}}}
\newcommand{\DataTypeTok}[1]{\textcolor[rgb]{0.13,0.29,0.53}{#1}}
\newcommand{\DecValTok}[1]{\textcolor[rgb]{0.00,0.00,0.81}{#1}}
\newcommand{\DocumentationTok}[1]{\textcolor[rgb]{0.56,0.35,0.01}{\textbf{\textit{#1}}}}
\newcommand{\ErrorTok}[1]{\textcolor[rgb]{0.64,0.00,0.00}{\textbf{#1}}}
\newcommand{\ExtensionTok}[1]{#1}
\newcommand{\FloatTok}[1]{\textcolor[rgb]{0.00,0.00,0.81}{#1}}
\newcommand{\FunctionTok}[1]{\textcolor[rgb]{0.00,0.00,0.00}{#1}}
\newcommand{\ImportTok}[1]{#1}
\newcommand{\InformationTok}[1]{\textcolor[rgb]{0.56,0.35,0.01}{\textbf{\textit{#1}}}}
\newcommand{\KeywordTok}[1]{\textcolor[rgb]{0.13,0.29,0.53}{\textbf{#1}}}
\newcommand{\NormalTok}[1]{#1}
\newcommand{\OperatorTok}[1]{\textcolor[rgb]{0.81,0.36,0.00}{\textbf{#1}}}
\newcommand{\OtherTok}[1]{\textcolor[rgb]{0.56,0.35,0.01}{#1}}
\newcommand{\PreprocessorTok}[1]{\textcolor[rgb]{0.56,0.35,0.01}{\textit{#1}}}
\newcommand{\RegionMarkerTok}[1]{#1}
\newcommand{\SpecialCharTok}[1]{\textcolor[rgb]{0.00,0.00,0.00}{#1}}
\newcommand{\SpecialStringTok}[1]{\textcolor[rgb]{0.31,0.60,0.02}{#1}}
\newcommand{\StringTok}[1]{\textcolor[rgb]{0.31,0.60,0.02}{#1}}
\newcommand{\VariableTok}[1]{\textcolor[rgb]{0.00,0.00,0.00}{#1}}
\newcommand{\VerbatimStringTok}[1]{\textcolor[rgb]{0.31,0.60,0.02}{#1}}
\newcommand{\WarningTok}[1]{\textcolor[rgb]{0.56,0.35,0.01}{\textbf{\textit{#1}}}}
\usepackage{graphicx}
\makeatletter
\def\maxwidth{\ifdim\Gin@nat@width>\linewidth\linewidth\else\Gin@nat@width\fi}
\def\maxheight{\ifdim\Gin@nat@height>\textheight\textheight\else\Gin@nat@height\fi}
\makeatother
% Scale images if necessary, so that they will not overflow the page
% margins by default, and it is still possible to overwrite the defaults
% using explicit options in \includegraphics[width, height, ...]{}
\setkeys{Gin}{width=\maxwidth,height=\maxheight,keepaspectratio}
% Set default figure placement to htbp
\makeatletter
\def\fps@figure{htbp}
\makeatother
\setlength{\emergencystretch}{3em} % prevent overfull lines
\providecommand{\tightlist}{%
  \setlength{\itemsep}{0pt}\setlength{\parskip}{0pt}}
\setcounter{secnumdepth}{-\maxdimen} % remove section numbering
\ifLuaTeX
  \usepackage{selnolig}  % disable illegal ligatures
\fi
\IfFileExists{bookmark.sty}{\usepackage{bookmark}}{\usepackage{hyperref}}
\IfFileExists{xurl.sty}{\usepackage{xurl}}{} % add URL line breaks if available
\urlstyle{same} % disable monospaced font for URLs
\hypersetup{
  hidelinks,
  pdfcreator={LaTeX via pandoc}}

\author{}
\date{\vspace{-2.5em}}

\begin{document}

\begin{Shaded}
\begin{Highlighting}[]
\NormalTok{monthly\_income = c(33, 24, 48, 32, 55, 74, 23, 17) \#Given}
\NormalTok{y \textless{}{-} monthly\_income}
\NormalTok{n \textless{}{-} length(y)}
\NormalTok{mu \textless{}{-} 3.6 \#Given}
\NormalTok{tau\_2 \textless{}{-} (sum((log(y) {-} mu)\^{}2) / n) \#Given}
\NormalTok{nDraws \textless{}{-} 10000 \#Given}
\NormalTok{draw\_inv\_chi \textless{}{-} function(nDraws, v, s)}
\NormalTok{\{}
\NormalTok{    draws \textless{}{-} rchisq(nDraws, v)}
\NormalTok{    return( (v * s)/draws )}
\NormalTok{\}}
\NormalTok{sigma2 \textless{}{-} draw\_inv\_chi(nDraws = nDraws, v = n, s = tau\_2)}

\NormalTok{plot(density(sigma2), xlim = c(0,1),}
\NormalTok{     main = "Density function of sigma2")}

\NormalTok{\#\#\#\#\#\#\#\#\#\#\#\#\#\#\#\#\#\#\#\#\#\#\#\#\#\#\#\#\#\#\#\#\#\#\#\#\#\#\#\#\#\#\#\#\#\#\#\#\#\#\#\#\#\#\#\#\#\#\#\#\#\#\#\#}
\NormalTok{\#b)}
\NormalTok{\#pnorm gives the cummulativ distribution function (phi(x))}
\NormalTok{CDF \textless{}{-} pnorm( sqrt(sigma2) / sqrt(2) )}
\NormalTok{G \textless{}{-} ((2 * CDF) {-} 1)}
\NormalTok{plot(density(G), main = "Density funciton for Gini{-}Coef")}


\NormalTok{\#\#\#\#\#\#\#\#\#\#\#\#\#\#\#\#\#\#\#\#\#\#\#\#\#\#\#\#\#\#\#\#\#\#\#\#\#\#\#\#\#\#\#\#\#\#\#\#\#\#\#\#\#\#\#\#\#\#\#\#\#\#\#\#}
\NormalTok{\#c)}
\NormalTok{lower \textless{}{-} quantile(G, 0.025)}
\NormalTok{upper \textless{}{-} quantile(G, 0.975)}

\NormalTok{plot(density(G), main = "Density funciton for Gini{-}Coef")}
\NormalTok{abline(v = lower, col = "red")}
\NormalTok{abline(v = upper, col = "red")}
\NormalTok{legend("topright", legend = c("Gini{-}distri", "Quantiles"),}
\NormalTok{       col = c("black", "red"), lty = 1)}
\NormalTok{\#\#\#\#\#\#\#\#\#\#\#\#\#\#\#\#\#\#\#\#\#\#\#\#\#\#\#\#\#\#\#\#\#\#\#\#\#\#\#\#\#\#\#\#\#\#\#\#\#\#\#\#\#\#\#\#\#\#\#\#\#\#\#\#}
\NormalTok{\#d)}
\NormalTok{\#\#\#\#\#\#\#\#\#\#\#\#\#\#\#\#\#\#\#\#\#\#\#\#\#\#\#\#\#\#\#\#\#\#\#\#\#\#\#\#\#\#\#\#\#\#\#\#\#\#\#\#\#\#\#\#}
\NormalTok{\#\# COULD BE DONE LIKE THIS}
\NormalTok{\#\# library("bayestestR")}
\NormalTok{\#\# interval = eti(G,0.95)}
\NormalTok{\#\#}
\NormalTok{\#\# sorted\_G = sort(G)}
\NormalTok{\#\# test = hdi(sorted\_G, ci = 0.95)}
\NormalTok{\#\#}
\NormalTok{\#\# AND PLOTTED LIKE THIS}
\NormalTok{\#\# plot(density(sorted\_G), col = "blue",xlim=c(0,.8))}
\NormalTok{\#\# abline(v= interval[2], col = "red")}
\NormalTok{\#\# abline(v= interval[3], col = "red")}
\NormalTok{\#\# abline(v= test[2], col = "green")}
\NormalTok{\#\# abline(v= test[3], col = "green")}
\NormalTok{\#\# OR LIKE BELOW}
\NormalTok{\#\#\#\#\#\#\#\#\#\#\#\#\#\#\#\#\#\#\#\#\#\#\#\#\#\#\#\#\#\#\#\#\#\#\#\#\#\#\#\#\#\#\#\#\#\#\#\#\#\#\#\#\#\#\#\#}
\NormalTok{kernel \textless{}{-} density(G)}
\NormalTok{kernel \textless{}{-} data.frame(kernel$y, kernel$x)}
\NormalTok{\#sort the DF based on the Density. }
\NormalTok{kernel \textless{}{-} kernel[order(kernel$kernel.y, decreasing = TRUE),]}
\NormalTok{tot\_sum \textless{}{-} sum(kernel$kernel.y)}
\NormalTok{cum\_dens \textless{}{-} 0.00000000000001}
\NormalTok{i \textless{}{-} 0}
\NormalTok{while ((cum\_dens/tot\_sum) \textless{} 0.95)}
\NormalTok{\{}
\NormalTok{  i \textless{}{-} (i + 1)}
\NormalTok{  cum\_dens \textless{}{-} cum\_dens + kernel$kernel.y[i]}
\NormalTok{\}}

\NormalTok{abline(v = max(kernel$kernel.x[1:i]), col = "blue")}
\NormalTok{abline(v = min(kernel$kernel.x[1:i]), col = "blue")}

\NormalTok{legend("topright", legend = c("Gini{-}distri", "Quantiles", "HPDI"),}
\NormalTok{       col = c("black", "red", "blue"), lty = 1)}
\end{Highlighting}
\end{Shaded}


\end{document}
